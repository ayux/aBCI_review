\section{Introduction}

% AFFECTIVE COMPUTING GENERAL IDEA

% PRIMARY RESPONSE TO ODORS, LINKS WITH EMOTIONS AND AFFECTIVE COMPUTING
%Olfactory perception can rise different primary responses in emotions. The rised emotional responses differ significantly between individuals with variations in pleasantness. Different subjects may give different emotional responses to the same odor. The mechanism of the variation between different individuals in perception of odors is still unclear. Different approaches in studying odor perception have been proposed by many groups both in physiological point of view~\cite{galizia2000odour} and genetic point of view~\cite{keller2007genetic}.
Perception of pleasantness from various stimuli has been investigated by various researchers through different means. Many studies have been conducted on the investigation of pleasantness perception through facial expressions~\cite{lyons1998coding}, food intake\cite{de2003taste}, languages~\cite{bellezza1986words}, etc. Research on pleasantness detection and classification has been carried out by analyzing brain activity using various brain imaging techniques(e.g.,~\cite{zatorre2000neural,kringelbach2003activation,kroupi2014eeg}).


Since multimedia systems are increasingly becoming immersive by rendering more realistic user-experience, which makes it better in evoking strong emotions. To study the emotion reactions to differnt multimedia contents can help us better understand, recognize and interpret the emotion responses to them, and further more can help us to simulate human affects to multimedia contents. Traditionally, multimedia systems include video and audio contents, they, thus, mainly stimulate the visual and auditory senses. Nevertheless, recently odors have started to be incorporated into multimedia systems (e.g., \cite{nakamoto2011olfactory,nakamoto2008cooking,richard2006multi}), since they directly stimulate memories and elicit strong emotions. However, emotion elicitation from odors has not been adequately investigated, although the primary response to smell is related to pleasantness perception \cite{gulas1995right}. 


% PREVIOUS WORKS/PAPERS ON ANALYSIS FROM PSD, BANDS FEATURES
Although pleasantness perception has been thoroughly analysed for various, especially audiovisual, stimuli, it has received less attention during experience of odors. Various Electroencephalography (EEG) studies on investigating odor pleasantness have analysed brain activation in terms of power spectral density features in frequency domain. To better understand and classify the pleasantness during odor perceiving, more information on how olfactory perception can affect emotions need to be discovered. To contribute to this study, we proposed to investigate the functional connectivity patterns in remote brain locations during olfactory perception process. Our hypothesis is that there are differences in the functional connectivity patterns when subjects experience pleasant and unpleasant odors. 
%%++++++++++++++++++++

% STRUCTURE OF THE PAPER
In order to validate our hypothesis, we designed experiments to record EEG signals during olfactory perception. Subjects' feedbacks on each ordor were self-reported after the perception process. Functional connectivity patterns of recorded EEG signals were calculated. To test if different connectivity patterns exist between pleasantness and unpleasantness, we extracted features from functional connectivity patterns and trained classifiers with these features. The evaluation of the classifier is subject-independent. Details on the methods we used are explained in the rest of this paper. Section II explains the background concept of functional connectivity and the concept of network feature of functional connectivity patterns, which will be used for pleasantness classification. Section III describes the experimental protocols and methods for constructing and measuring functional connectivity maps from EEG signals. Section IV provides the classification results on pleasantness by using network-based features extracted from the functional connectivity patterns. Section V gives the conclusions of this work. 