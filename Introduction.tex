\section{Introduction}

% AFFECTIVE COMPUTING GENERAL IDEA

% PRIMARY RESPONSE TO ODORS, LINKS WITH EMOTIONS AND AFFECTIVE COMPUTING
%Olfactory perception can rise different primary responses in emotions. The rised emotional responses differ significantly between individuals with variations in pleasantness. Different subjects may give different emotional responses to the same odor. The mechanism of the variation between different individuals in perception of odors is still unclear. Different approaches in studying odor perception have been proposed by many groups both in physiological point of view~\cite{galizia2000odour} and genetic point of view~\cite{keller2007genetic}.
Perception of pleasantness from various stimuli has been investigated by various researchers through different means. Many studies have been conducted on the investigation of pleasantness perception through facial expressions~\cite{lyons1998coding}, food intake\cite{de2003taste}, languages~\cite{bellezza1986words}, etc. Research on pleasantness detection and classification has been carried out by analyzing brain activity using various brain imaging techniques(e.g.,~\cite{zatorre2000neural,kringelbach2003activation,kroupi2014eeg}).


Since multimedia systems are increasingly becoming immersive by rendering more realistic user-experience, which makes it better in evoking strong emotions. To study the emotion reactions to differnt multimedia contents can help us better understand, recognize and interpret the emotion responses to them, and further more can help us to simulate human affects to multimedia contents. Traditionally, multimedia systems include video and audio contents, they, thus, mainly stimulate the visual and auditory senses. Nevertheless, recently odors have started to be incorporated into multimedia systems (e.g., \cite{nakamoto2011olfactory,nakamoto2008cooking,richard2006multi}), since they directly stimulate memories and elicit strong emotions. However, emotion elicitation from odors has not been adequately investigated, although the primary response to smell is related to pleasantness perception \cite{gulas1995right}. 


% PREVIOUS WORKS/PAPERS ON ANALYSIS FROM PSD, BANDS FEATURES
Although pleasantness perception has been thoroughly analysed for various, especially audiovisual, stimuli, it has received less attention during experience of odors. Various Electroencephalography (EEG) studies on investigating odor pleasantness have analysed brain activation in terms of power spectral density features in frequency domain. 

%%++++++++++++++++++++
To better understand and classify the pleasantness during odor perceiving, more information on how olfactory perception can affect emotions need to be discovered. To contribute to this study, we propose to study the functional connectivity patterns in remote brain locations during olfactory perception process. Our hypothesis is that there are differences in the functional connectivity patterns when subjects experience pleasant and unpleasant odors. 
%%++++++++++++++++++++

 but further information from EEG could equally or better contribute to a deeper understanding of pleasantness perception from odors. For instance, a question that still remains unanswered is regarding the way in which odor pleasantness influences functional connectivity patterns in remote brain locations. The hypothesis is that there are differences in the functional connectivity patterns when subjects experience pleasant and when they experience unpleasant odors. 

%The emotion status during odor perception still need to be investigated for a better understanding the olfactory perception and emotion generation processes. The current studies on classifying emotional responses from odors based on EEG signals are more focused on extracting features of power spectral density, workload index (WI), permutation entropy, dimension of minimal covers~\cite{kroupi2014eeg}. 

% VALUE OF FUNCTIONAL CONNECTIVITY OF EEG
%Most of the methods that have been proposed and commonly used in EEG-based emotion study are performed directly on EEG signals themselves, while in this study we propose to use an indirect method to study about these EEG signals which is called \emph{functional connectivity map}. 
In general, there are different levels of connectivity in the brain, ranging from the connections between neighbouring neurons to the connections between different segregated brain regions. The concept of functional connectivity describes the dependence between the activities of different neuron assemblies. The functional connectivity map describes the dependency between all the sub-regions of brain (all the electrodes in EEG signals) regardless of whether they are structurally linked or not. The concept of functional connectivity maps \emph{functional connectivity map} has been largely used in neuro-imaging analysis (EEG, fMRI, PET and etc), to study major depression~\cite{greicius2007resting} and epilepsy~\cite{waites2006functional} disorders with respect to resting-states. Based on the neurophysiological principle~\cite{van2010exploring}, brain can be viewed as a complex network. Thus in this paper, we will introduce the concept of functional connectivity map in order to understand the networking of the brain during odor perception.

% OUR METHODS, NO RESULTS
Since the dependence between different sub-regions of the brain can be interpreted in different ways, there are different models of estimating the functional connectivity map of the brain. Widely used models include \emph{Granger Causality}, \emph{Transfer Entropy}, \emph{Nonlinear Regression Analysis} and \emph{Spectral Coherence}. In this paper, we applied and compared three different models in estimating functional connectivity maps which can also give us a better view of what is happening during the processes of odour perception and emotion generation. 

% STRUCTURE OF THE PAPER
The rest of the paper is organised as follows: Section II describes the experimental protocols and methods for constructing and measuring functional connectivity maps from EEG signals. Section III provides the classification results on pleasantness by using the network based features extracted from functional connectivity maps. Section IV gives the conclusion of this work. 