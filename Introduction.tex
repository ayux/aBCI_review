\section{Introduction}

% PRIMARY RESPONSE TO ODORS, LINKS WITH EMOTIONS AND AFFECTIVE COMPUTING
Emotion is involved in every aspect of human life, thus, it has gained attention in many research disciplines, including computer science. For instance, multimedia systems are increasingly becoming immersive, in order to evoke strong emotions and render user-experience more realistic. Traditionally, multimedia systems include video and audio contents, they, thus, mainly stimulate the visual and auditory senses. Nevertheless, recently odors have started to be incorporated into multimedia systems (e.g., \cite{nakamoto2011olfactory,nakamoto2008cooking,richard2006multi}), since they directly stimulate memories and elicit strong emotions. However, emotion elicitation from odors has not been adequately investigated, although the primary response to smell is related to pleasantness perception \cite{gulas1995right}. 

% RESEARCHER'S INVESTIGATIONS ON PLEASANTNESS
Perception of pleasantness from various stimuli has been investigated by various researchers through different means. Many studies have been conducted on the investigation of pleasantness perception through facial expressions~\cite{lyons1998coding}, food intake\cite{de2003taste}, languages~\cite{bellezza1986words}, etc. Research on pleasantness detection and classification has been carried out by analyzing brain activity, especially using Positron Emission Tomography (PET)~\cite{zatorre2000neural} and functional Magnetic Resonance Imaging (fMRI)~\cite{kringelbach2003activation}. EEG approaches in emotion detection and recognition have been also conducted by various research groups to classify the emotional responses of pleasantness from various affective stimuli (e.g., ).

% PREVIOUS WORKS/PAPERS ON ANALYSIS FROM PSD, BANDS FEATURES
Although pleasantness perception has been thoroughly analysed for various, especially audiovisual, stimuli, it has received less attention during experience of odors. Moreover, various EEG studies on investigating odor pleasantness have analysed brain activation in terms of power spectral density features, but further information from EEG could equally or better contribute to a deeper understanding of pleasantness perception from odors. For instance, a question that still remains unanswered is regarding the way in which odor pleasantness influences functional connectivity patterns in remote brain locations. The hypothesis is that there are differences in the functional connectivity patterns when subjects experience pleasant and when they experience unpleasant odors. 

In general, there are different levels of connectivity in the brain, ranging from the connections between neighbouring neurons to the connections between different segregated brain regions. The concept of functional connectivity describes the dependence between the activities of different neuron assemblies. The functional connectivity map describes the dependency between all the sub-regions of brain (all the electrodes in EEG signals) regardless of whether they are structurally linked or not. The concept of functional connectivity maps has been largely used in neuro-imaging analysis (EEG, fMRI, PET and etc), to study major depression~\cite{greicius2007resting} and epilepsy~\cite{waites2006functional} disorders with respect to control states. Based on the neurophysiological principle~\cite{van2010exploring}, brain can be viewed as a complex network. Thus in this paper, we will introduce the concept of functional connectivity map to understand the networking of the brain during odor perception.

% OUR METHODS, NO RESULTS
Since the dependence between different sub-regions of the brain can be interpreted in different ways, there exists different models of estimating the functional connectivity map of the brain. Widely used models include \emph{Granger Causality}~\cite{granger1969investigating}, \emph{Transfer Entropy}~\cite{schreiber2000measuring}, \emph{Nonlinear Regression Analysis}~\cite{pijn1990localization} and \emph{Spectral Coherence}~\cite{sun2004measuring}. In this paper, we applied and compared two different models in estimating functional connectivity maps. To have a better view of what is happening during the processes of odour perception and emotion generation, we also applied network-based features to estimated functional connectivity maps including \emph{characteristic path}~\cite{watts1998collective}, \emph{local and global efficiency}, \emph{clustering coefficient}~\cite{latora2001efficient}, \emph{shannon entropy} and \emph{von Neumann entropy}~\cite{passerini2008neumann}. The extracted features were then used for classification of the pleasantness of odor perception, which is also a target of this paper. 

% STRUCTURE OF THE PAPER
The rest of the paper is organised as follows: Section II describes the experimental protocols and methods for constructing and measuring functional connectivity maps from EEG signals. Section III provides the classification results on pleasantness by using the network based features extracted from functional connectivity maps. Section IV gives the conclusion of this work. 