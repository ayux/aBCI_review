\section{Background}

\subsection{Functional Connectivity}
Brain connectivity refers to a pattern of anatomical links ("anatomical connectivity") or of statistical dependencies (functional connectivity) between neural assemblies. The connectivity pattern is formed by structural links such as synapses or represented by statistical or calsal relationships measured as cross-correlation, coherence or information flow~\cite{sporns2007brain}. Brain connectivity is a crucial concept to elucidate how neural networks process information. 
A neurophysiological concept of functional connectivity is introduced by A.A Fingelkurts~\cite{fingelkurts2005functional}. According to Fingelkurts' concept, functional connectivity is described as the mechanism for the coordination of activity between different neural assemblies in order to achieve a complex cognitive task or perceptual process. However, since the dependence between different sub-regions of the brain can be interpreted in different ways, there exist different models of estimating functional connectivity in the brain. Widely used models include time-domain estimations such as \emph{Linear Granger Causality}~\cite{granger1969investigating}, \emph{Nonlinear Regression Analysis}~\cite{pijn1990localization}, \emph{Transfer Entropy}~\cite{schreiber2000measuring} and frequency-domain estimations such as \emph{Spectral Coherence}~\cite{sun2004measuring}.

The concept of functional connectivity has been largely used in neuro-imaging analysis in the study of major depression~\cite{greicius2007resting} and epilepsy~\cite{waites2006functional}. To our knowledge, little work has been done on emotion recognization on olfactory perception by analysing functional connectivity patterns. Thus in this paper, we introduce the concept of functional connectivity to investigate the connctivity patterns between brain regions during odor pleasantness perception. We applied and compared two time-domain functional connectivity models: \emph{Linear Granger Causality} and \emph{Nonlinear Regression Analysis}. By comparing the performances of the two models, we can better understand the arrangement of connections between brain regions during olfactory perception and emotion generation. The differences of connectivity patterns between pleasant and unpleasant ordor perceptions can be generalized to other emotion study and also help us process or stimulate human affects in the future work. 

\subsection{Network Features}
The functional connectivity gives us a view of how channels communicate information with each other, thus we can consider the whole connectivities over brain as a kind of brain network. For an N-channel EEG signal recording, functional connectivity estimates the connectivity between each channel combination, which results in an $N \times N$ connectivity network. With the increase of number of channels $N$, the brain network will become larger and more difficult to be analyzed directly. Thus, we proposed to extract network-based features from the brain network. The extracted network-based features from original functional connectivity over brain can provide a higher-level view on the characteristics of the brain network. 

Some research groups see the brain network as a kind of small-world network~\cite{bassett2006small} while some others view it as a scale-free network~\cite{eguiluz2005scale}. Both scale-free network and small-world network can provide features based on their own characteristics. Small-world network can provide features as \emph{characteristic path}~\cite{watts1998collective}, \emph{local and global efficiency}, \emph{clustering coefficient}~\cite{latora2001efficient}. Scale-free network can provide features as \emph{shannon entropy} and \emph{von Neumann entropy}~\cite{passerini2008neumann}. In this paper, we extracted both types of features for training the classifiers. The performances of the classifiers can help us understand which type of network the functional connectivity network can be considered as. 

